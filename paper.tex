
%%%%%%%%%%%%%%%%%%%%%%%%%%%%%%%%%%%%%%%%%%%%%%%%%%%%%%%%%%%%%%%%%%
% = TO COMPILE THIS DOCUMENT =
%
% From the command line, it would go like this --- assuming you are
%    in the directory where the filename.tex source file and the 
%    filename.bib bibliography file are located, and that you have 
%    permission to create and write files in that directory:
%      > pdflatex filename
%      > bibtex filename
%      > pdflatex filname
%      > pdflatex filename
%    Yes, you run the command several times. The earlier runs 
%    create auxilliary files which keep track of references,
%    citations, equation and section numberring, etc. The later
%    runs combine the information in these auxilliary files with
%    your source document to create the finished product.
%
% If you are using a GUI LaTeX editor like TeXWorks, then there
%    is probably a menu bar button for pdfLaTeX and another for
%    BibTeX. Hit them in the order indicated above. There is 
%    probably also a 'TeXify' button, or something similarly named,
%    which runs all the above commands in one shot.     
%%%%%%%%%%%%%%%%%%%%%%%%%%%%%%%%%%%%%%%%%%%%%%%%%%%%%%%%%%%%%%%%%%


%%%%%%%%%%%%%%%%%%%%%%%%%%%%%%%%%%%%%%%%%%%%%%%%%%%%%%%%%%%%%%%%%%
%  = PREAMBLE =
% The preamble of a LaTeX document is the set of commands that precede
% the \begin{document} line.  It contains a \documentclass line
% to load the REVTeK-4.1 macro definitions and various \usepackage
% lines to load other macro packages.
%
% ADVICE TO STUDENTS: This preamble contains a suggested set of
%     class options to generate a ``Junior Lab'' look and feel that
%     facilitate quick review and feedback from one's peers, TAs,
%     and section instructors.  Don't make substantial changes 
%     to the style without first consulting your section 
%     instructor.
%%%%%%%%%%%%%%%%%%%%%%%%%%%%%%%%%%%%%%%%%%%%%%%%%%%%%%%%%%%%%%%%%%

%\documentclass[aps,twocolumn,secnumarabic,balancelastpage,amsmath,amssymb,nofootinbib, floatfix]{revtex4}
\documentclass[aps,twocolumn,secnumarabic,balancelastpage,amsmath,amssymb,nofootinbib,floatfix]{revtex4-1}


%%%%%%%%%%%%%%%%%%%%%%%%%%%%%%%%%%%%%%%%%%%%%%%%%%%%%%%%%%%%%%%%%%%
% = Explanation of documentclass options =
%
% aps, prl stand for American Physical Society and Physical 
%     Review Letters respectively.
% twocolumn permits two columns, of course.
% nobalancelastpage doesn't attempt to equalize the lengths of 
%     the two columns on the last page  as might be desired in a 
%     journal where articles follow one another closely.
% amsmath and amssymb are necessary for the subequations 
%     environment among others. These functionalities can
%     also be added use the usepackage function described below,
%     but REVTeX conveniently includes them as documentclass
%     options.
% secnumarabic identifies sections by number to aid electronic 
%     review and commentary.
% nofootinbib forces footnotes to occur on the page where they are
%      first referenced and not in the bibliography.
% floatfix attempts to help LaTeX decide where to place ``floats'',
%      like figures and plots, when it gets stuck and can't decide
%      by it's normal algorithm.
% REVTeX 4.1 is a set of macro packages designed to be used with 
%      LaTeX 2e. REVTeX is well-suited for preparing manuscripts 
%      for submission to APS journals.
%
% = Other documentclasses =
%
% The 'revtex4' and 'revtex4-1' documentclasses are somewhat 
%    specialized for making documents in the style of the APS
%    journals. For a more standard or generic looking LaTeX paper,
%    you could try any of the built-in documentclasses, in 
%    particular 'article' or 'report'. Someday, you may try to use 
%    the 'mitthesis'  documentclass available for download from the 
%    MIT Libraries. The vast majority of source code written for 
%    one documentclass should work just fine in any other, but 
%    occasional quirks arise. For example, some documentclasses 
%    disagree on whether the abstract declaration should come 
%    before or after the \begin{document} declaration.
% 
%%%%%%%%%%%%%%%%%%%%%%%%%%%%%%%%%%%%%%%%%%%%%%%%%%%%%%%%%%%%%%%%%%%

%% Now, include some packages which provide new commands that 
%% extend LaTeX's capabilities. Note that the nearly-essential
%% AMS math packages were added already as documentclass options
%% for REVTeX, but could have been added here using 
%% \usepackage{amsmath}, etc. The pacakges below are commonly 
%% useful, but there are many, many more available to solve a 
%% multitude of typesetting quandries (google your problem), 
%% and you  probably have the necesary packages installed on your
%% system already. Among the examples listed below, this sample
%% document only actually makes use of the 'graphicx', 'bm', 
%% and 'hyperref' pacakges, so the others are commented out for
%% tidyness.


\usepackage{graphicx}      % tools for importing graphics
%\usepackage{lgrind}        % convert program code listings to a form 
                            % includable in a LaTeX document
%\usepackage{xcolor}        % produces boxes or entire pages with 
                            % colored backgrounds
%\usepackage{longtable}     % helps with long table options
%\usepackage{epsf}          % old package handles encapsulated postscript issues
\usepackage{bm}            % special bold-math package. usge: \bm{mathsymbol}
%\usepackage{asymptote}     % For typesetting of mathematical illustrations
%\usepackage{thumbpdf}
\usepackage[colorlinks=true]{hyperref}  % this package should be added after 
                                        % all others.
                                        % usage: \url{http://web.mit.edu/8.13}


%%%%%%%%%%%%%%%%%%%%%%%%%%%%%%%%%%%%%%%%%%%%%%%%%%%%%%%%%%%%%%%%%%%
% And now, begin the document...
%%%%%%%%%%%%%%%%%%%%%%%%%%%%%%%%%%%%%%%%%%%%%%%%%%%%%%%%%%%%%%%%%%%

\begin{document}
\title{Polarization topology of Bound States in Photonic Slabs.
\author{Francisco Leal Machado}
\email{fmachado@mit.edu}
%\homepage{} %If you don't have one, just comment out this line.
\date{\today}
\affiliation{MIT Department of Physics}

\begin{abstract}
WRITE Abstract
\end{abstract}

\maketitle

\section{Introduction}

\section{Symmetry considerations}

\section{Previous Work}

\section{New Geometries}

\bibliography{paper}


\end{document}
